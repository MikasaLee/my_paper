\section{Introduction}
计算机断层扫描(COMPUTED tomography, CT)由于能够在不破坏物体的情况下实现物体的内部视觉,已广泛应用于临床、工业和其他领域\cite{2018Deep}。但CT的电离辐射会对人体造成危害\cite{2012Hall},这严重限制了它的实际应用。在临床诊断中,为了去降低辐射剂量和减少扫描时间而采用sparse view CT。然而,投影视图的缺陷给重建的图像带来ill-posed inverse problems\cite{1989Incomplete}。许多重建算法被提出用来解决这些问题,他们一般被分为3类:(a) sinogram domain pre-processing,(b) iterative algorithm, and (c)image domain post-processing.\par

sinogram domain pre-processing首先对sinograms 进行上采样和去噪,然后再将它们重建出CT图像。在过去的几十年中,在singram domain中提出了非线性平滑\cite{2004Nonlinear}、结构自适应滤波\cite{2012Ray}和基于字典学习的图像修补方法\cite{2014Dictionary}方法用于上采样和去噪。然后再用经典的解析算法(en:classic analytical method)如filtered back-projection (FBP)\cite{kak2001principles}重建出CT图像。然而,由于重建对 sinogram domain中产生的误差很敏感,这些方法的性能往往会受到影响。\par

除了简单的back projection和改进的FBP算法,在过去的几十年中,图像重建更多使用的是iterative algorithm。尤其是将压缩感知(CS)\cite{2006Robust}\cite{2006Donoho}引入到迭代重建之后,CT图像质量大幅度提升。其中最著名的是总变异(TV)\cite{2008Image}。除此之外,iterative algorithm还包括 nonlocal means (NLM)\cite{2009Bayesian},tight wavelet frames\cite{2011Multi},dictionary learning\cite{2012Low,2019Convolutional},low rank\cite{2014Cine} 以及TV之后的改进版\cite{2014Sparse,2016Statistical,2013Few}。然而,上述迭代重建方法由于计算量巨大以及难以调优的参数,导致其需要较长的计算时间以及很难去泛化不同的扫描方案或人体部位产生的不同图像。\par

最后一类稀疏视图CT重建算法是image domain post-processing。稀疏视图CT投影数据经过解析算法(en:classic analytical method,如FBP)重构后,CT图像中会出现噪声和条纹伪影等问题。image domain post-processing方法通过去除解析算法重建之后的这些问题来提高CT图像的质量。受稀疏表示理论的启发,Aharon等人将字典学习\cite{2006ksvd}应用于LDCT去噪,显著提高了腹部图像的质量。Han\cite{2016Sparse}等人考虑到光谱中的其他信息,提出low-rank Hankel matrix(ALOHA)从sparse-view中重建出CT图像。与其他两种方法相比,噪声在图像域中的分布不能准确确定,这使得post-processing方法无法在保持结构和噪声替代之间实现最优的权衡。随着计算机硬件性能的提升以及深度学习的快速发展促成了许多在医学图像重建领域的成功应用。例如,Chen等人\cite{redcnn}提出了一种残差的编码器-解码器卷积神经网络(Red-CNN)来从FBP重建中去除伪像。Jin等人\cite{2016FBPConvNet}提出了一种结合了FBP、u-net和残差学习的深度卷积网络(deep convolutional network, FBPconvNet),在保留图像结构的同时去除CT图像的伪影。Eunhee等人\cite{8332971}结合小波变换和深度残差学习解决低剂量CT降噪,这篇文章的方法赢得了2016 Low-Dose CT Grand Challenge第二名。为了使预测图像服从与NDCT相同的统计分布,Wolterink等人\cite{2017Generative}中引入生成对抗网络(generative adversarial network, GAN),并利用判别器网络实现该约束。Yang等人\cite{2018Low}提出使用带有Wasserstein距离和感知损失的生成式对抗网络(WGAN-GP)从FBP重建的CT图像中恢复微妙的结构。Chen等人\cite{2018LEARN}展开了最陡梯度下降算法,提出了稀疏视图CT的基于学习专家评估的重构网络(LEARN)。尽管这些方法已经显示出良好的效果,然而CNN中图像内部的长期依赖性(如图像中对象的非局部相关性)被忽略,导致图像的部分信息丢失。\par

近年来,为了解决CNN感受野有限的问题,一种基于注意力的编码器-解码器结构-Transformer\cite{vaswani2017attention}被提出来,并在许多计算机视觉任务中取得了巨大成功。Dosovitskiy等人\cite{dosovitskiy2020vit}提出了Vision Transformer(ViT)用于图像分类,这是Transformer首次用于计算机视觉,并在著名的ImageNet数据集\cite{2009ImageNet}上突破了SOTA。Liu等人\cite{liu2021Swin}提出了通过滑动窗口来弥补patch边缘信息的Swin Transformer,这也是本文主要的灵感来源。Chen等人提出了IPT\cite{chen2021pre},这是transformer第一次应用于low-level计算机视觉任务。Wang等人提出了一种基于U-net架构的Transformer模型\cite{wang2021uformer}用于图像重建。Luthra等人\cite{2021Eformer}提出基于边缘增强的transformer模型,用于医学图像去噪。尽管以上Transformer模型在其他计算机视觉的任务上表现优异,然而据我们所知,目前还没有将Transformer用于在sparse-view下的双域CT重建任务中。\par

In this paper,我们提出了一个全新的模型-DDPTransformer用于解决sparse-view双域CT重建,利用transformer相比与CNN获得的全局信息的优势,我们超越了现有的最先进的方法并且证明了transformer对于双域重建的有效性。DDPTransformer包括了4步,首先,用插值法模拟正常采样下得到的Sinogram图,其次我们对模拟正常采样的Sinogram通过Sinogram Domain SubNet进行优化,Then通过FBP将sinogram重建出CT图,但得到的CT图像充满条纹状的伪影和噪声,最后通过Image Domain SubNet进行去除噪声和伪影,得到高质量的CT图像。我们的贡献总结如下:

\begin{itemize} 
		\item 我们提出了一种新的sinogram域和image域联合学习框架用于解决sparse-view CT重建。该模型在不同view以及不同器官上展现出了优异的性能,这证明了我们模型的鲁棒性。
		\item 我们设计了一个全新的基于Transformer的backbone模块:DDPTransformer Block。它有效解决了基于卷积所带来的感受野有限的问题,并同时通过并行Transformer的方式来互补Patch的边缘信息。
		\item 与在CV中基本的Transformer模型(如ViT)不同的是,考虑到我们输入的是2-D图片,所以我们设计了Layer-Conv-Layer(LCL)模块来代替Transformer中的MLP进行特征提取,解决因降维带来丢失部分信息的问题。并在实验部分我们证明了我们LCL模块的有效性。
\end{itemize}

本文的其余部分组织如下。In Section II,我们详细阐述了DDPTransformer的实现细节。大量的实验结论in Section III。In Section IV,我们将进一步讨论模型的鲁棒性。the remarks on conclusions and future works are presented in Section V。\par