\section{Conclusions}
CT技术已成为现代医学中最常用的辅助诊断手段。为了避免CT电离辐射对人体造成的损伤,稀疏采样下的应用越来越受到重视。为了从sparse-view的sinogram重建出高质量的CT图像,本文提出了一个基于Transformer的模型用于解决sparse-view双域CT重建,我们通过Transformer来替代卷积用来解决感受野有限的问题,并通过并行的方式解决了Patch块的边缘信息难以学习的问题,以及通过Layer-Conv-Layer(LCL)模块来代替MLP,使得Transformer在获得长距离可以学习更多的图像信息。实验结果表明该模型适用于不同的sparse-view以及不同的器官组织,与其他重建算法相比在PSNR,SSIM,RMSE等不同的评价指标上均取得了最好的结果。并在去噪去伪影的同时有效保留CT图像的结构和纹理信息,得到令人满意的视觉效果。最后我们在其他数据集上对其鲁棒性进行了进一步验证,结果表明在不同的数据集上其效果依然是最好的。