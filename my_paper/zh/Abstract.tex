\noindent{\bf Abstract: }对于临床诊断来说,CT技术是必不可少的。但由于CT使用有害的电离辐射,所以每做一次CT会对人体造成不可逆的损伤。为了减少辐射对人体的影响,研究人员通过减少采样点的个数得到稀疏采样图像,但稀疏采样下的图像会带来图像不清晰,条纹伪影严重等问题,从而为诊断带来不好的影响。我们的研究内容是从稀疏采样下的sinogram中恢复高质量的CT图像。近年来基于深度学习的方法被广泛应用于CT重建,然而长期以来,基于CNN的模型由于感受野通常很小,所以在处理大尺寸的图像上往往会丢失全局信息。为了解决这个问题,我们提出了一种基于Transformer的模型(called:DDPTransformer)来获取全局信息并在"\emph{Low Dose CT Image and Projection Data (LDCT-and-Projection-data)}"数据集上进行训练和验证模型性能,更具体的说,我们使用Transformer来代替传统的卷积操作,并通过不同的切Patch方式使得两个Transformer可以互相拟补Patch块的边缘信息。结果显示在不同的稀疏采样下模型均表现出优异的性能,和其他先进的算法对比有了很大提高。并最后通过在不同的数据集上验证了模型的鲁棒性。代码和模型可以在此公开获得:xxxxx\\

\noindent{\bf Keywords: }Deep learning;sparse view CT reconstruction;Transformer;dual domains